\section{Questions and answers}
\subsection*{Question 1}

what of the following remarks regarding TLS are possibly true?
\begin{itemize}
  \correct has been established a session S1 that involved
    connection C1 and C2 at time t1
  \correct has been established a session S2 that involved
    connection C1, C2 and C3, at different and non- overlapping time
    intervals
  \incorrect connection C5 has been used initially in session S3
    and then re-used in session S4 for performance-sake
  \incorrect in connection C1.1 of session S1 has been used
    AES192, while in Connection C1.2 of the same session S1 has been
    used AES128 since regarded less sensible information
\end{itemize}

\subsection*{Question 2}
focusing on TLS Connections and Sessions
\begin{itemize}
  \correct sessions are typically relatively long-life in respect to
  connections
  \incorrect connections allow to re-use cryptographic parameters
  defined during handshake, thus reducing significantly the handshake
  phase
  \incorrect by using resumption mechanisms, the client can resume a
  connections decreasing the overall connection time
  \correct each connection has its own specific set of parameters like
  seguence numbers and keys for integrity and confidentiality
\end{itemize}
\subsection*{Question 3}
focusing on Perfect Forward Secrecy
\begin{itemize}
  \correct starting from version TLS version 1.3, it must be always
  enabled
  \incorrect using RSA mechanisms in TLS, it would be theoretically
  impossible to achieve
  \correct using RSA mechanisms in TLS, it would be impractical to
  achieve
  \incorrect using ECDH mechanisms in TLS, it would be impractical due
  to the length of the key parameters
\end{itemize}
\subsection*{Question 4}
Which of the following properties is NOT directly related to Perfect
Forward Secrecy in a TLS context?
\begin{itemize}
  \incorrect the use of ephemeral keys 
  \correct protection against replay attacks
  \incorrect security of past sessions if the server's private key is compromised
  \incorrect independent session keys for each connection
\end{itemize}

\subsection*{Question 5}
In the context of TLS 1.3, what is the role of PFS in session
resumption mechanisms like O-RTT?
\begin{itemize}
  \incorrect O-RTT resumption is vulnerable to replay attacks, thus
  does not support PFS
  \incorrect O-RTT session resumption uses pre-shared keys that
  provide Perfect Forward Secrecy.
  \correct O-RTT session resumption compromises Perfect Forward
  Secrecy for faster handshakes.
  \incorrect O-RTT session resumption reuses the original session's
  ephemeral keys
\end{itemize}


\subsection*{Question 6}
Which TLS feature was introduced to specifically mitigate downgrade attacks?
\begin{itemize}
  \incorrect HSTS (HTTP Strict Transport Security) 
  \incorrect OCSP Stapling
  \incorrect Certificate Pinning
  \correct TLS FALLBACK SCSV
\end{itemize}


\subsection*{Question 7}
Given the following packet capture:
\begin{itemize}
  \incorrect indicate a 
  \incorrect might be the initial phase of a secured real-time data
  exchange (like video streaming)
  \incorrect It refers to an aborted TLS session, due to the
  impossibility to verify the x509v3 certificate
  \correct lt refers to an aborted TLS session, due to TLS version
  mismatch between the versions supported by the server and by the
  client
\end{itemize}



